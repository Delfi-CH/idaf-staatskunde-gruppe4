\documentclass[a5paper]{article}

\usepackage[left=1.25cm, right=1.25cm, bottom=2.5cm, top=2cm]{geometry}
\usepackage[ngerman]{babel}
\usepackage[T1]{fontenc}
\usepackage[utf8]{inputenc}

\usepackage{fancyhdr}

\usepackage{caption}
\captionsetup[figure]{labelformat=empty}

\usepackage{float}

\usepackage[
    backend=biber,
    style=numeric,
    sorting=nyt
]{biblatex}
\setlength\bibitemsep{1.0\baselineskip}
\addbibresource{quellen.bib}

\usepackage{wrapfig}
\usepackage{graphicx}
\graphicspath{ {./images/} }

\setlength{\headsep}{12.5mm}

\title{Einführung Legislative}
\author{Matteo Krüger \\ Mathis Fritz \\ Laurens Spitzer}
\date{02.12.2025}

\pagestyle{fancy}
\fancyhf{}
\fancyhead[L]{02.12.2025}
\fancyhead[R]{\includegraphics[width=0.25\textwidth]{kbw.jpg}}
\fancyfoot[L]{\small{Kantonsschule \\ Büelrain} }
\fancyfoot[C]{\thepage}
\fancyfoot[R]{\small{IDAF - Gruppe 4 \\ Staatskunde}}

\begin{document}
    \maketitle
    \tableofcontents
    \newpage
    \section{Einführung} %1
        In dieser Broschüre werden die Aufgaben und Zuständigkeiten der Schweizer Legislative, unterteilt in zwei Kammern, den Nationalrat und den Ständerat, genauer erläutert.
    \vspace{0.5cm}
    \section{Rechtsetzung} %2
        Eine der Hauptaufgaben der Legislative ist die Gesetzgebung. Zusammen entscheiden die insgesamt 246 demokratisch gewählten Mitglieder der Bundesversammlung über Gesetzesänderungen auf nationaler Ebene. \\ 
        Dies gilt ebenfalls für Änderungen der Bundesverfassung, wobei die Kontrolle darüber stärker ist, weil ein doppeltes Mehr (Mehrheit im Volk und der Kantone) zwingend notwendig ist.
        \subsection{Komissionen} %2.1
            Das Parlament besitzt auch Komissionen, welche aus einigen Ratsmitgliedern bestehen. Hauptsächlich beraten sie und arbeiten Vorschläge für das Parlament aus. In diesem Kapitel werden einige davon exemplarisch beleuchtet.
            \subsubsection{Legislaturplannungskommissionen LPK} %2.1.1
                Die LPK bestehen aus 22 Mitgliedern des Nationalrates und 11 Mitgliedern des Ständerates (Stand 2025). Sie gehören zu den Spezialkommissionen und ihre Aufgabe besteht darin, den Entwurf des Bundesrates zur Legislaturplanung vorzuberaten. Dieser Entwurf wird zu Beginn jeder Legislaturperiode, also nach jeder Gesamterneuerungswahl der Räte, der Bundesversammlung vorgelegt. Er definiert politische Ziele und plant Erlasse der Bundesversammlung zur entsprechenden Umsetzung.
            \subsubsection{Geschäftsprüfungskommissionen GPK} %2.2.2
                Die GPK bestehen aus 24 Mitgliedern des Nationalrates und 13 Mitgliedern des STänderates (Stand 2025). Sie gehören zu den Aufsichtskomissionen und überprüfen die Geschäftsführung des Bundesrates und der Bundesverwaltung. 
    \vspace{0.5cm}
    \section{Finanzen} %3
        Die Bundesversammlung entscheidet nicht nur über Gesetze, sondern auch über die Finanzen des Bundes. Dies umfasst hauptsächlich den Voranschlag sowie etwaige Nachtragskredite.
        \subsection{Voranschlag} %3.1
            Das Parlament nimmt einen Vorschlag des Bundesrates entgegen und ist befugt, Änderungen zu erlassen. Der Voranschlag zeigt für das kommende Jahr und die drei darauffolgenden Jahre die geplanten Ausgaben und Einnahmen des Bundes.
        \subsection{Nachtragskredite} %3.2
            Sollte sich im Laufe des Jahres zeigen, dass die zur Verfügung gestellten Mittel nicht ausreichen, kann ein Nachtragskredit für einzelne Sektoren beantragt werden.
            \subsubsection{Ordentliches Verfahren} %3.2.1
                Zweimal jährlich legt der Bundesrat der Bundesversammlung die Nachtragskredite vor. Diese werden in der Sommer- oder Wintersession behandelt.
            \subsubsection{Dringliches Verfahren} %3.2.2
                Sollte es einen Ausgabe oder Investition geben, welche nicht aufgeschoben werden kann, ist der Bundesrat dazu berechtigt, den Nachtragskredit mit Zustimmung der Finanzdelegation selbst zu beschliessen. Die Zustimmung der Bundesversammlung muss jedoch nachträglich eingeholt werden.
    \vspace{0.5cm}
    \section{Wahl der Bundesorgane} %4
        Eine weitere Aufgabe des Parlaments ist die Wahl der Bundesorgane, also der Mitglieder der obersten Bundesbehörden.  Dazu gehören unter anderem der Bundesrat (4 Jahre Amtszeit) zusammen mit dem Bundeskanzler (1 Jahr Amtszeit) oder der Bundeskanzlerin, die Richterinnen und Richter der Bundesgerichte (6 Jahre Amtszeit), sowie die Bundesanwaltschaft (4 Jahre Amtszeit).
        \subsection{Bundesratswahlen} %4.1
            Die Mitglieder des Bundesrates werden von der Bundesversammlung auf eine Amtszeit von 4 Jahren gewählt. Die Gesamterneuerungswahlen geschehen in der Wintersession nach den Nationalratswahlen, traditionsgemäss am Mittwoch der zweiten Sessionswoche. Die letzte Gesamterneuerungswahl (Stand 2025) war am 13. Dezember 2023. Die neu gewählten Mitglieder treten ihr Amt am 1. Januar nach der Wahl an. \\
            Ausserordentliche Gesamterneuerungswahlen sind durch Totalrevisionen der Verfassung möglich. Diese können vom Volk oder von den Räten initiiert werden. Wird einer Totalrevision zugestimmt, kommt es zu ausserordentlichen Neuwahlen des Bundesrates sowie der beiden Kammern der Bundesversammlung. Jeder Bundesrat ist Vorstand eines Departementes. Diese Verteilung geschieht nicht durch das Parlament.
            \vspace{0.75cm}
            \begin{figure}[H]
                \begin{center}
                    \includegraphics[width=0.8\textwidth]{a.o._Gesamterneuerung-d.png}
                    \caption{Ablauf Gesamterneuerungswahlen \cite{ao_gesamterneuerung}}
                \end{center}
            \end{figure}
    \newpage
    \section{Oberaufsicht} %5
        Weiter übernehmen die Aufsichtskomissionen der Räte in deren Auftrag die Oberaufsicht über die Organe des Bundes. Dies umfasst im wesentlichen vier Punkte:
        \begin{itemize}
            \item Die Übereinstimmung von Entscheiden der Bundesbehörden mit der Verfassung
            \item Die Zweckmässigkeit der von den Behörden gewählten Massnahmen
            \item Die Wirksamkeit jener Massnahmen sowie
            \item Das Verhältnis zwischen eingesetzten Ressourcen und erhaltenem Ergebnis
        \end{itemize}
        Zur Durchsetzung dieser Kontrolle führen die Aufsichtskomissionen Inspektionen durch oder besichtigen die Amtsstellen der Verwaltung. Auch werden die Staatsrechnung, der Geschäftsbericht des Bundesrates und weitere Jahresberichte der Organe überprüft. \\
        Die Informationssuche der Aufsichtsorgane beschränkt sich jedoch nicht auf Bundesbehörden, sondern umfasst auch Personen und Amtsstellen ausserhalb der Bundesverwaltung, sofern diese Informationen besitzen, welche für die Oberaufsicht notwendig sind.
    \vspace{0.5cm}
    \section{Mitwirken in der Aussenpolitik} %6
        Auch die Aussenpolitik der Schweiz wird massgeblich vom Parlament beeinflusst. Der Bundesrat steht in Kontakt mit den für die Aussenpolitik zuständigen Kommissionen der Bundesversammlung. Grundsätzlich sind völkerrechtliche Verträge von der Bundesversammlung abzusegnen. Dies wird \textit{ordentliches Verfahren} genannt. \\ 
        Im Gegensatz dazu existiert das \textit{vereinfachte Verfahren}, bei welchen der Bundesrat zum selbstständigen Vertragsabschluss ermächtigt ist. 
    \newpage
        \subsection{Ordentliches Verfahren} %6.1
            \begin{wrapfigure}{r}{0.35\textwidth}
                \includegraphics[width=0.9\linewidth]{Genehmigung_Staatsvertraege-d.png}
                \caption{Ordentliches Verfahren \cite{staatsvertraege}}
            \end{wrapfigure}
            Der Bundesrat befragt die für die Aussenpolitik zuständigen Kommissionen zu den Richtlinien und führt die Vertragsverhandlugnen. Nach der Unterzeichnung wird der Vertrag an die Bundesversammlung zur Genehmigung weitergereicht.  \\
            Grundsätzlich wird nur über den gesamten Vertrag entschieden. Ist dies jedoch möglich, kann das Parlament den Bundesrat zur Einbringung eines Vorbehaltes verpflichten. Wird ein Vertrag zum zweiten Mal abgelehnt, ist die Entscheidung entdgültig. \\
            Hat die Bundesversammlung dem Vertrag zugestimmt, wird dieser vom Bundesrat ratifiziert und in Kraft gesetzt. 
        \subsection{Vereinfachtes Verfahren} %6.2
            Ein Gesetz oder oder ein durch das Parlament genehmigter völkerrechtlicher Vertrag kann der Bundesrat selbstständig einen Vertrag abschliessen. \\
            Das Parlament erhält einen jährlichen Bericht vom Bundesrat über die abgeschlossenen Verträge. Sehen sich die Räte als zuständig, können sie ein nachträgliches ordentliches Verfahren verlangen.
    \vspace{0.5cm}
    \section{Gewährleistung der Verfassungen} %7
        Die Bundesversammlung ist ebenfalls für die Gewährleistung der Kantonsverfassungen verantwortlich. Sie überprüft den Inhalt der Verfassungen und entscheidet, ob diese der Auslegung der Bundesverfassung entspricht. Die Kantone müssen jegliche Verfassungsänderungen mitteilen und um eine Überprüfung bitten.
    \printbibliography
\end{document}