\documentclass[a5paper]{article}

\usepackage[left=1.25cm, right=1.25cm, bottom=2.5cm, top=2cm]{geometry}
\usepackage[ngerman]{babel}
\usepackage[T1]{fontenc}
\usepackage[utf8]{inputenc}

\usepackage{fancyhdr}

\usepackage{float}

\usepackage{graphicx}
\graphicspath{ {./images/} }

\setlength{\headsep}{12.5mm}

\title{Einführung Legislative}
\author{Matteo Krüger \\ Mathis Fritz \\ Laurens Spitzer}
\date{02.12.2025}

\pagestyle{fancy}
\fancyhf{}
\fancyhead[L]{02.12.2025}
\fancyhead[R]{\includegraphics[width=0.25\textwidth]{kbw.jpg}}
\fancyfoot[L]{\small{Kantonsschule \\ Büelrain} }
\fancyfoot[C]{\thepage}
\fancyfoot[R]{\small{IDAF - Gruppe 4 \\ Staatskunde}}

\begin{document}
    \maketitle
    \newpage
    \section{Einführung} %1
        In dieser Broschüre werden die Aufgaben und Zuständigkeiten der Schweizer Legislative, unterteilt in zwei Kammern, den Nationalrat und den Ständerat, genauer erläutert.
    \section{Rechtsetzung} %2
        Eine der Hauptaufgaben der Legislative ist die Gesetzgebung. Zusammen entscheiden die insgesamt 246 demokratisch gewählten Mitglieder der Bundesversammlung über Gesetzesänderungen auf nationaler Ebene. \\ 
        Dies gilt ebenfalls für Änderungen der Bundesverfassung, wobei die Kontrolle darüber stärker ist, weil ein doppeltes Mehr (Mehrheit im Volk und der Kantone) zwingend notwendig ist.
\end{document}